% !TEX root = editor-tfp16.tex

%% Programs (and, by the Curry-Howard correspondence, proofs) are rich
%% inductive structures. This fact is well understood amongst researchers
%% and experienced programmers, but still somewhat obscure amongst
%% programmers at large because programmers normally construct and interact
%% with programs only indirectly, e.g. using a text editor composed with a
%% parser.

%% There are some benefits to this approach, to be sure, but the structural
%% mismatch between programs and their textual representations also imposes
%% various burdens.  For example, the primitive edit actions available in a
%% text editor (e.g. inserting or deleting a character or word) do not
%% always correspond to sensible structural transformations.

% spj: describe the problem; state our contributions; STOP. one page max.

When constructing a program or proof in a language with rich type
structure, skilled programmers generally follow a \emph{type discipline}
where they first determine the type of the expression that they are
constructing in order to constrain the mental search space that they are
operating within.

For example, if the programmer knows that an expression of type
$\tarr{\tnum}{\tnum}$ is needed, then it is often the case (though, of course, not
necessarily the case) that the expression will take the form $$\hlam{\mathit{x}}{e}$$
for some variable $x$ and function body $e$. If the programmer chooses this
form, then after picking a suitable variable name, her focus will be on
constructing a suitable body, $e$. Following the type discipline, $e$ must
be of type $\tnum$, and so this process can begin anew.

The problem is that when using a text
editor to construct a program, it is easy to deviate from this disciplined process.  Rather, text editors operate on sequences of
characters (i.e. \emph{text}.) Although programs can be represented as text, not
every sequence of characters corresponds to a well-formed program, much
less a semantically well-defined program. Programmers and the tools that they use are thus 
often left to examine and manipulate text for which there are no useful
reasoning principles, either because they are in the midst of a sequence of
text editor actions that leave the program ``temporarily'' malformed or ill-defined, or because the programmer has made a mistake.

% stuck editing a representation of the program instead of the structures
% themselves. The editor does not restrict what the programmer may do: you
% can delete characters that belong, insert ones that don't, forget things
% that were needed, and so. There's nothing stopping us from accidentally
% writing $$\lambda \mathit{x:num}.\mathit{(x,x)}$$ even though it's obvious
% that building a pair can't hope to form a natural number.

% The type structure of the language makes this sort of error obvious: it's
% not that you're adding characters that make your program incorrect, or even
% malformed; you're adding characters that can't possibly create a structure
% you want because of the type. Simply put, the primitive operations
% available in text editors do not always correspond to sensible
% transformations on the structure of the program.

\emph{Structure editors} have shown some progress toward addressing this
problem by allowing programmers to edit the tree structure of a program
directly (we give examples of notable structure editors in Sec. \ref{sec:rw}.) Each editor action leaves the program being constructed in a
structurally well-formed state. However, this maintains only structural
reasoning principles, i.e. those that tools like syntax highlighters rely
on. It is still possible to leave the program in a semantically undefined state.

%% TODO: some sort of transition that talks about how you still don't have
%% semantic reasoning principles because you can be left in a semantically
%% ill-defined state

In this paper, we present our ongoing work towards a small structure editor, Hazelnut,  defined in the type-theoretic style. In Hazelnut, every edit action leaves the program in both a structurally and statically well-defined state.
\begin{itemize}
  \item We begin with an example to develop the reader's intuitions in Section
    \ref{sec:example}.

  \item We then formally define Hazelnut in Section \ref{sec:hazel}. This
    lambda calculus introduces an explicit notion of terms with holes in
    them to represent programs that are not yet complete.
    (Section \ref{sec:holes}) Programs may have many holes in them, and
    holes are acted upon with respect to their type and which holes are
    currently in focus. (Sections \ref{sec:cursors} and \ref{sec:actions})

    We also state two key metatheoretic properties this type theory
    enjoys. (Section \ref{sec:mt})

  \item We describe our ongoing effort to formalize the metatheory of this
    calculus, including the proofs of the theorems
    above. (Section \ref{sec:mech})

  \item We describe our ongoing work to connect this type theory with a
    usable implementation of a real structure editor as a web
    application. (Section \ref{sec:impl})

  \item We conclude by describing our intentions for this work going
    forward. (Section \ref{sec:future})
\end{itemize}
