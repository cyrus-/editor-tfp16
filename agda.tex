In the previous section, we gave an overview of the most important rules in
the semantics of Hazelnut, and stated the important theorems. In a few
cases, we sketched out why these theorems will hold.

In order to formally verify that our design meets the stated objectives, we
are preparing a formalization of the grammar, the judgements given above
and the metatheory in the proof assistant Agda \cite{norell:thesis}. We
refer readers unfamiliar with the particulars of Agda to the Wiki, hosted
at \url{http://wiki.portal.chalmers.se/agda/}.

This formalization is a work in progress at the time of submission, but we
plan on completing it by the time this work is presented. The goal is fully
mechanized proofs Theorem \ref{thrm:actsafe} and Theorem \ref{thrm:actdet}
that act as a platform to explore more sophisticated designs and
metatheory. The development can be found at
\url{http://github.com/hazelgrove/agda-tfp16}, which includes the actual Agda
source and a more detailed discussion of the technical representation
decisions made.

The core idea of our formalization is to encode each judgement as a
dependent type. The rules of the judgements become the constructors of the
type, and derivations of theorems values of the type. This is a rich
setting that allows proofs to take advantage of pattern matching on the
shape of derivations, closely matching on-paper proofs of similar
properties.

The formalization differs from the precise calculus listed here in a few
ways. Most interestingly, instead of giving separate rules for the movement
of each form of H-expressions and corresponding Z-expressions, we abstract
over different corresponding forms that happen to have the same arity. This
formalizes the observation that the rules for movements on additions and
movements on applications are morally the same. Formalizing this intuition
reduces the number of cases we need to consider in the inductive
considerably, but more importantly allows us to write a slightly more
general calculus. The rules will work for any form of a particular arity,
so it's much more straightforward to extend this calculus to work for more
interesting language features.

Because the metatheory in this paper is largely concerned with the statics
of Hazelnut rather than the dynamics, we adopt Barendregt's convention for
bound variables and avoid substitution entirely.\cite{urban} Future
iterations will need a more mature technique for reasoning about
binding---likely de Bruijn indices or abstract bounding
trees.\cite{lh09unibind} \cite{Pouillard11} By avoiding these issues when
they are not yet relevant, we offer a clear first-pass formalization that
hews closely to the mathematics presented in the paper.
